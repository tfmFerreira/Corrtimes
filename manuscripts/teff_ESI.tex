\documentclass[journal=jpcbfk,manuscript=article,layout=twocolumn]{achemso}
\usepackage{amsmath}
\usepackage{amssymb}
%\usepackage{widetext}
\usepackage{verbatim}
\usepackage{graphicx}
% \usepackage{multicol}
\usepackage[normalem]{ulem} % for strikethrough
\usepackage[obeyFinal]{easy-todo}
\usepackage{dutchcal}
\usepackage{xfrac}
\usepackage{footmisc}

\newcommand{\figurewidth}{.48\textwidth}
\renewcommand{\epsilon}{\varepsilon}
\newcommand{\dz}{\,\mathrm{d}z}
\graphicspath{{Figures/}}
\newcommand{\onlinecite}[1]{\hspace{-1 ex} \nocite{#1}\citenum{#1}}
\SectionNumbersOn
%\DeclareUnicodeCharacter{2009}{FIXME}

\author{Hanne S. Antila}
\affiliation{Department of Theory and Bio-Systems, Max Planck Institute of Colloids and Interfaces, 14424 Potsdam, Germany}

\author{Tiago Ferreira}
\affiliation{NMR Group --- Institute for Physics, Martin-Luther University Halle--Wittenberg, 06120 Halle (Saale), Germany}

\author{Matti Javanainen}
\affiliation{Add Matti to author list?}

\author{O. H. Samuli Ollila}
\affiliation{Institute of Biotechnology, University of Helsinki, 00014 Helsinki, Finland}

\author{Markus S. Miettinen}
\affiliation{Department of Theory and Bio-Systems, Max Planck Institute of Colloids and Interfaces, 14424 Potsdam, Germany}
\email{markus.miettinen@iki.fi}

%\title{Using open data to benchmark internal dynamics of phosphatidylcholine in molecular dynamics simulations}
\title{ESI: Using open data to rapidly bench\-mark bio\-molecular simulations: Phospholipid internal dynamics}
\begin{document}




\section{Experimental setup and analysis of $R_1$, $R_{1\rho}$ and $\tau_{\rm{e}}$}
All experiments were conducted on a standard bore E-free CP-MAS 4mm probe, using a Bruker AVANCE II-500 NMR spectrometer, operating at a $^{13}$C Larmor frequency of 125.78 MHz. The magic angle spinning (MAS) rate used was 5 kHz and the effective temperature was 298 K. Sample preparation and the $R_1$ and $R_{{\rm{1}}\rho}}$ experiments were as described in reference \cite{ferreira15}.\\

The following pages show the fits performed to determine the relaxation rates shown in figures x and y of the main manuscript. The error estimates for $R_1$ and $R_{1\rho}$ were defined as the highest deviation between the 95\% confidence bounds and the fitted values of $R_1$ and $R_{1\rho}$ using MATLAB 2018b~\cite{} and the curve fitting toolbox package to fit single exponential decays of the form $a\exp(-R_1 t)$ (figures x-y). Each point of the relaxation decays for each individual $^{13}$C peak was calculated by analytical integration of a fitted gaussian lineshape to the  $^{13}$C peak.      



\bibliography{lipids,ff,simu,pdb,journals}



\end{document}
